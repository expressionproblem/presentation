\documentclass[10pt, compress]{beamer}

\usetheme[usetitleprogressbar]{m}

\usepackage{booktabs}
\usepackage[scale=2]{ccicons}
\usepackage{tcolorbox}
\tcbuselibrary{minted}

\usepgfplotslibrary{dateplot}

\usemintedstyle{monokai}

\definecolor{backg}{RGB}{28, 28, 28}
\tcbset{boxrule=0mm,colback=backg,colframe=backg,arc=0mm,listing only,listing engine=minted,minted options={fontsize=\scriptsize}}
\title{Extensibility for the masses}
\subtitle{Practical Extensibility with Object Algebras}
\date{\today}
\author{Jorn van Wijk 3718778}
\institute{Universiteit Utrecht}

\begin{document}

\maketitle

\begin{frame}[fragile]{The expression problem}
Extension of data abstractions subject to:
\begin{enumerate}
\item Extensibility in both dimensions 
	\begin{itemize}
		\item new data variants
		\item new operations
	\end{itemize}
%A solution must allow the addition of new data variants and new operations and support extending existing operations. 
\item Strong static type safety % A solution must prevent applying an operation to a data variant which it cannot handle using static checks.
\item No modification or duplication
% Existing code must not be modified nor duplicated.
\item Separate compilation and type-checking
% Safety checks or compilation steps must not be deferred until link or runtime.
\item Independent extensibility
%It should be possible to combine independently developed extensions so that they can be used jointly.
\end{enumerate}

Formulated by Philip Wadler (1-4)\\
Extended by Zenger and Odersky (5)
\end{frame}

\begin{frame}[fragile]{The expression problem}{Functional languages}
Functional programming languages
can easily add new operations
\begin{tcblisting}{minted language=haskell}
data Expr = Const Int
          | Add Expr Expr
		  
eval x = ..
numConst = ..
prettyPrint = ..
\end{tcblisting}

But adding new data variants is hard.
\end{frame}
\begin{frame}[fragile]{Example OO code}
Like the following:
\begin{tcblisting}{minted language=java}
interface Exp {
    Value eval();
}
class Lit implements Exp {
    int x;
    public Lit(int x) { this.x = x; }
    public Value eval() {
        return new VInt(x);
}}
class Add implements Exp {
    Exp l, r;
    public Add(Exp l, Exp r) { this.l = l; this.r = r; }
    public Value eval() {
        return new VInt(l.eval().getInt() + r.eval().getInt());
}}
\end{tcblisting}

\end{frame}

\section{Elements}

\begin{frame}[fragile]
  \frametitle{Typography}
      \begin{minted}[fontsize=\small]{latex}
The theme provides sensible defaults to \emph{emphasize}
text, \alert{accent} parts or show \textbf{bold} results.
      \end{minted}

  \begin{center}becomes\end{center}
  
\begin{minted}[mathescape,
               linenos,
               numbersep=5pt,
               gobble=2,
               frame=lines,
               framesep=2mm]{csharp}
  string title = "This is a Unicode π in the sky"
  /*
  Defined as $\pi=\lim_{n\to\infty}\frac{P_n}{d}$ where $P$ is the perimeter
  of an $n$-sided regular polygon circumscribing a
  circle of diameter $d$.
  */
  const double pi = 3.1415926535
\end{minted}

  The theme provides sensible defaults to \emph{emphasize} text,
  \alert{accent} parts or show \textbf{bold} results.
\end{frame}
\begin{frame}{Lists}
  \begin{columns}[onlytextwidth]
    \column{0.5\textwidth}
      Items
      \begin{itemize}
        \item Milk \item Eggs \item Potatos
      \end{itemize}

    \column{0.5\textwidth}
      Enumerations
      \begin{enumerate}
        \item First, \item Second and \item Last.
      \end{enumerate}
  \end{columns}
\end{frame}
\begin{frame}{Descriptions}
  \begin{description}
    \item[PowerPoint] Meeh.
    \item[Beamer] Yeeeha.
  \end{description}
\end{frame}
\begin{frame}{Animation}
  \begin{itemize}[<+- | alert@+>]
    \item \alert<4>{This is\only<4>{ really} important}
    \item Now this
    \item And now this
  \end{itemize}
\end{frame}
\begin{frame}{Figures}
  \begin{figure}
    \newcounter{density}
    \setcounter{density}{20}
    \begin{tikzpicture}
      \def\couleur{mLightBrown}
      \path[coordinate] (0,0)  coordinate(A)
                  ++( 90:5cm) coordinate(B)
                  ++(0:5cm) coordinate(C)
                  ++(-90:5cm) coordinate(D);
      \draw[fill=\couleur!\thedensity] (A) -- (B) -- (C) --(D) -- cycle;
      \foreach \x in {1,...,40}{%
          \pgfmathsetcounter{density}{\thedensity+20}
          \setcounter{density}{\thedensity}
          \path[coordinate] coordinate(X) at (A){};
          \path[coordinate] (A) -- (B) coordinate[pos=.10](A)
                              -- (C) coordinate[pos=.10](B)
                              -- (D) coordinate[pos=.10](C)
                              -- (X) coordinate[pos=.10](D);
          \draw[fill=\couleur!\thedensity] (A)--(B)--(C)-- (D) -- cycle;
      }
    \end{tikzpicture}
    \caption{Rotated square from
    \href{http://www.texample.net/tikz/examples/rotated-polygons/}{texample.net}.}
  \end{figure}
\end{frame}
\begin{frame}{Tables}
  \begin{table}
    \caption{Largest cities in the world (source: Wikipedia)}
    \begin{tabular}{lr}
      \toprule
      City & Population\\
      \midrule
      Mexico City & 20,116,842\\
      Shanghai & 19,210,000\\
      Peking & 15,796,450\\
      Istanbul & 14,160,467\\
      \bottomrule
    \end{tabular}
  \end{table}
\end{frame}
\begin{frame}{Blocks}

  \begin{block}{This is a block title}
    This is soothing.
  \end{block}

\end{frame}
\begin{frame}{Math}
  \begin{equation*}
    e = \lim_{n\to \infty} \left(1 + \frac{1}{n}\right)^n
  \end{equation*}
\end{frame}
\begin{frame}{Line plots}
  \begin{figure}
    \begin{tikzpicture}
      \begin{axis}[
        mlineplot,
        width=0.9\textwidth,
        height=6cm,
      ]

        \addplot {sin(deg(x))};
        \addplot+[samples=100] {sin(deg(2*x))};

      \end{axis}
    \end{tikzpicture}
  \end{figure}
\end{frame}
\begin{frame}{Bar charts}
  \begin{figure}
    \begin{tikzpicture}
      \begin{axis}[
        mbarplot,
        xlabel={Foo},
        ylabel={Bar},
        width=0.9\textwidth,
        height=6cm,
      ]

      \addplot plot coordinates {(1, 20) (2, 25) (3, 22.4) (4, 12.4)};
      \addplot plot coordinates {(1, 18) (2, 24) (3, 23.5) (4, 13.2)};
      \addplot plot coordinates {(1, 10) (2, 19) (3, 25) (4, 15.2)};

      \legend{lorem, ipsum, dolor}

      \end{axis}
    \end{tikzpicture}
  \end{figure}
\end{frame}
\begin{frame}{Quotes}
  \begin{quote}
    Veni, Vidi, Vici
  \end{quote}
\end{frame}


\section{Conclusion}

\begin{frame}{Summary}

  Get the source of this theme and the demo presentation from

  \begin{center}\url{github.com/matze/mtheme}\end{center}

  The theme \emph{itself} is licensed under a
  \href{http://creativecommons.org/licenses/by-sa/4.0/}{Creative Commons
  Attribution-ShareAlike 4.0 International License}.

  \begin{center}\ccbysa\end{center}

\end{frame}

\plain{Questions?}

\end{document}
